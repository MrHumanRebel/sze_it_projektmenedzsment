%----------------------------------------------------------------------------
\chapter{\projectlifecycle}
%----------------------------------------------------------------------------

A projektmenedzsment szakirodalom ötfázisú életciklust ír le: indítás, tervezés, megvalósítás, ellenőrzés és lezárás \cite{Hajdu2014,Szalay2018}. A TeDeRMS fejlesztését úgy szerveztem, hogy minden fázisban mérhető eredmény készüljön, és a gyakorlatban igazoljam az elméleti elveket.

\section{Projektindítás}

A kiinduló pont az igényfelmérés és a scope definiálása volt. A vállalatnál végzett interjúk és a korábbi projektek tapasztalatai alapján üzleti követelmény-listát készítettem, amelyet prioritás szerint rangsoroltam. Ez megfelel a szakirodalom által javasolt érintetti elemzésnek és üzleti célokhoz kötött célfa készítésének \cite{Kovacs2016,Kaposi2019}. A projektindító dokumentumban rögzítettem az elvárt átfutási időt (10 hónap) és a minőségi kritériumokat (adatpontosság, rendelkezésre állás), így később ezekhez mértem a teljesítést.

\section{Tervezés}

Az ütemezést Gantt-szerkezetben készítettem el: igényértelmezés (2 hét), architektúra-terv (2 hét), modulfejlesztés (16 hét), integrációs teszt (4 hét), bevezetés (2 hét). A mérföldkövekhez konkrét átadási tételeket rendeltem, hogy a controlling szakaszban objektív mérőpontok legyenek \cite{Hajdu2014}. Az erőforrás-terv az önálló fejlesztés miatt heti 25 munkaórával számolt, és tartalékidőt is tartalmazott a kockázatos modulokra (auth, jogosultság, riportok) \cite{Szalay2018}. A kockázatok előzetes feltárását kockázatmátrixba rögzítettem, amely a valószínűség-hatás párok alapján priorizált.

\begin{figure}[H]
    \centering
    \includegraphics[width=70mm, keepaspectratio]{figures/plan.png}
    \caption{Fő mérföldkövek időterve. \footnotesize Forrás: saját szerkesztés, projektterv összefoglaló}
    \label{fig:plan}
\end{figure}

\section{Megvalósítás}

A fejlesztés moduláris architektúrán alapult (auth, dashboard, eszközkezelés, projektek, riportok), ami lehetővé tette a párhuzamos, de kontrollált fejlesztést. A Git-flow szerinti ágkezelés és a konténerizált környezet biztosította, hogy minden módosítás reprodukálható legyen \cite{Kovacs2016}. A backend és frontend komponenseket iteratív sprintekben szállítottam: minden sprint végén működő, tesztelt modul készült, így a projekt több alkalommal produkált használható verziót, ahogy az iteratív modellek ajánlják \cite{Kaposi2019}.

\section{Ellenőrzés és irányítás}

Heti státusz-összefoglalót készítettem az előrehaladásról, amelyben a becsült és tényleges időráfordításokat hasonlítottam össze. Az eltérések esetén újraterveztem az ütemezést (pl. a jogosultsági modul több időt igényelt). A minőségbiztosítást unit- és funkcionális tesztekkel végeztem: kritikus út moduljai (auth, eszközkeresés) minden sprintben regressziós tesztet kaptak \cite{Szalay2018}. A monitorozás során azonosított hibákat issue-ként rögzítettem, és a kockázati szintjük alapján priorizáltam.

\section{Lezárás}

A bevezetés előtt pilotot futtattam a vállalat egyik kisebb projektjén, amely során valós adatokkal ellenőriztem a folyamatokat. A felhasználói visszajelzések alapján javítottam a naptárnézetet és az eszközszűrést. A lezárás részeként átadtam a rendszer- és üzemeltetési dokumentációt, valamint az üzemeltetési checklistát. A tapasztalatok összegzése a záró fejezetben található, követve a tudásmenedzsmentre vonatkozó ajánlásokat \cite{Hajdu2014}.
