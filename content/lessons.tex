%----------------------------------------------------------------------------
\chapter{\lessons}
%----------------------------------------------------------------------------

A TeDeRMS projekt során szerzett tapasztalatok bemutatják, hogyan kapcsolható össze a projektmenedzsment elmélet és a vállalati gyakorlat.

\section{Projektmenedzsment tanulságok}

\begin{itemize}
    \item \textbf{Célok és mérföldkövek:} a mérhető mérföldkövek (auth, eszközmodul, riportok) folyamatos visszajelzést adtak, és csökkentették a scope-kúszás kockázatát \cite{Hajdu2014}.
    \item \textbf{Rendszeres monitorozás:} a heti státusz-összefoglalók segítették a becslések kalibrálását, a kockázatos feladatokat pedig előrehozták a következő sprintekbe \cite{Szalay2018}.
    \item \textbf{Stakeholder bevonás:} a kulcsfelhasználókkal tartott demók biztosították, hogy az elméleti követelmények (RACI, felelősségi körök) valódi működési folyamatokká váljanak \cite{Kaposi2019}.
\end{itemize}

\section{Fejlesztési és technológiai tapasztalatok}

\begin{itemize}
    \item \textbf{Modularitás:} az önállóan fejlesztett modulok (auth, eszközkezelés, projektek) külön tesztkörnyezetet kaptak, ami gyors hibakeresést tett lehetővé és csökkentette a regressziókat \cite{Kovacs2016}.
    \item \textbf{Konténerizálás:} a Docker alapú környezet minimalizálta a „működik a gépemen” típusú problémákat, és biztosította a reprodukálható bevezetést \cite{Szalay2018}.
    \item \textbf{Dokumentáció:} a rendszer- és üzemeltetési dokumentáció elkészítése megfelelt a lezárási fázis tudásmegosztási elvárásainak, és a pilot után finomítottam őket a felhasználói visszajelzések alapján \cite{Hajdu2014}.
\end{itemize}

\begin{figure}[H]
    \centering
    \includegraphics[width=60mm, keepaspectratio]{figures/devops.png}
    \caption{A DevOps szemlélet szerepe a projektben. \footnotesize Forrás: saját szerkesztés, projekt tapasztalatok alapján}
    \label{fig:devops}
\end{figure}
