%----------------------------------------------------------------------------
\chapter{\riskproblem}
%---------------------------------------------------------------------------- 

A \textbf{TeDeRMS} projekt során a kockázatkezelés meghatározó szerepet játszott, 
mivel a projekt önálló fejlesztésként valósult meg, így minden döntés és probléma gyors és hatékony kezelést igényelt. 
A kockázatkezelés célja a potenciális problémák előrejelzése, azok hatásának minimalizálása, 
valamint a projekt sikerének biztosítása volt.

%----------------------------------------------------------------------------
\section{Felmerült problémák}
%----------------------------------------------------------------------------

A fejlesztés során számos kihívással és nehézséggel szembesültem, amelyek hatékony kezelése elengedhetetlen volt a projekt sikeres megvalósításához, 
valamint a rendszer megbízható és teljes funkcionalitásának biztosításához.
A tapasztalatok alapján megfigyelhető, hogy a projektmenedzsment klasszikus fázisai jól előre jelzik a 
problémák jellegét, de a gyakorlatban gyakran szükség van rugalmasságra és kreatív megoldásokra is \cite{Hajdu2014,Szalay2018}.

%----------------------------------------------------------------------------
\subsection{Technológiai problémák}
%----------------------------------------------------------------------------
A rendszer stabilitása és a Docker-konténerek kezdeti konfigurációja nem volt optimális, ami futtatási hibákhoz vezetett.
Ez a probléma rámutatott arra, hogy a modern szoftverfejlesztési környezetekben a technológiai infrastruktúra 
optimalizálása legalább olyan fontos, mint a szoftver logikájának megtervezése. 
A stabil technológiai alap kritikus a rendszer megbízhatósága szempontjából, 
és a kezdeti hibák megfelelő dokumentációval és iteratív konfigurálással kezelhetők \cite{Kovacs2016,Kaposi2019}. 
Az ilyen problémák korai felismerése jelentősen csökkentette a későbbi üzemeltetési nehézségeket.

%----------------------------------------------------------------------------
\subsection{Funkcionális kihívások}
%----------------------------------------------------------------------------

A bérléskezelési folyamatok pontos leképezése a szoftverben komoly kihívást jelentett, különösen a különböző státuszok és jogosultságok kezelése során.  
A funkcionalitás implementálása során világossá vált, hogy a vállalati folyamatok összetettsége gyakran túlmutat a papíron megadott szabályokon.  

A rendszertervezés során a folyamatok részletes feltérképezése és a valós működés elemzése fontos a hibák minimalizálásához.  
Megfigyelésem alapján a moduláris architektúra alkalmazása és a folyamatok iteratív tesztelése kell a pontos implementáció eléréséhez, biztosítva, hogy a rendszer a vállalat tényleges igényeit hűen tükrözze.

%----------------------------------------------------------------------------
\subsection{Időmenedzsment}
%----------------------------------------------------------------------------

Az önálló projektvégzés során különösen nehéz volt összehangolni a napi feladatokat a projekt folyamatos előrehaladásával.
Az időmenedzsment kérdése kiemelten fontos volt, mivel a korlátozott erőforrások és a napi feladatok összehangolása 
folyamatos priorizálást igényelt. A projekt időbeli koordinációja, 
a mérföldkövek betartása és a személyes produktivitás optimalizálása kritikus tényezők az önálló fejlesztések során \cite{Kaposi2019,Kovacs2016}. 
Értékelésem alapján a heti tervezés és a feladatok rendszeres felülvizsgálata segített a haladás fenntartásában.

%----------------------------------------------------------------------------
\subsection{Tesztelés és hibajavítás}
%----------------------------------------------------------------------------

A rendszer moduláris felépítése miatt a hibák lokalizálása és javítása több iterációt igényelt.
A hibák azonosítása és javítása a moduláris felépítés miatt komplex feladat volt, mivel egy-egy modul problémája 
láncreakcióként hatott a rendszer más részeire. A folyamatos unit- és integrációs 
tesztelés, valamint a hibák dokumentálása és visszacsatolása alapvető a projekt stabilitása szempontjából \cite{Hajdu2014,Szalay2018}. 
Személyes tapasztalatom szerint az iteratív hibajavítás és a részletes tesztelési protokollok alkalmazása biztosította 
a moduláris rendszer hatékony működését, miközben a megszerzett módszerek későbbi projektekben is hasznosíthatók
%----------------------------------------------------------------------------
\section{Kockázatok azonosítása és priorizálása}
%----------------------------------------------------------------------------

A projekt kezdeti fázisában kiemelten fontos volt a potenciális problémák előrejelzése és kezelése, 
mivel a kockázatok megfelelő azonosítása és priorizálása közvetlen hatással van a projekt sikerére 
és a rendszer minőségére. A kockázatmenedzsment alapja a kockázatok strukturált feltérképezése, 
az értékelésük és a megfelelő megelőző intézkedések kialakítása \cite{Hajdu2014,Szalay2018,Kovacs2016}.

%----------------------------------------------------------------------------
\subsection{Magas prioritású kockázatok}
%----------------------------------------------------------------------------

A magas prioritású kockázatok közé tartoznak a kritikus hibák a backend működésében, az adatvesztés és a biztonsági hiányosságok.  
Ezek a problémák a rendszer alapvető működését fenyegették, ezért azonnali figyelmet igényeltek.  
A magas prioritású problémák kezelése kulcsfontosságú a rendszer stabilitásának és megbízhatóságának biztosításához \cite{Kaposi2019,Szalay2018}.  
A kritikus hibák korai azonosítása és a preventív intézkedések bevezetése jelentősen csökkentette 
a rendszer működési kockázatát, és elősegítette a fejlesztés zavartalan folytatását.

%----------------------------------------------------------------------------
\subsection{Közepes prioritású kockázatok}
%----------------------------------------------------------------------------

A közepes prioritású kockázatok a felhasználói felület hibáit és a riportok pontosságát érintették.  
Bár ezek nem veszélyeztetik közvetlenül a rendszer alapvető működését, hosszú távon befolyásolhatják 
a felhasználói élményt és az elfogadottságot \cite{Hajdu2014,Kovacs2016}.  
Megítélésem szerint a közepes prioritású problémák folyamatos monitorozása és iteratív javítása hozzájárult a felhasználói élmény optimalizálásához, 
anélkül, hogy a kritikus funkciókat veszélyeztettük volna
%----------------------------------------------------------------------------
\subsection{Alacsony prioritású kockázatok}
%----------------------------------------------------------------------------

Az alacsony prioritású kockázatok főként a rendszer megjelenését vagy a jövőbeli bővítési igényeket érintették.  
Bár ezek kevésbé sürgetők, hosszú távon hozzájárulnak a rendszer karbantarthatóságához és a fejlesztési terv fenntarthatóságához \cite{Szalay2018,Kaposi2019}.  
Általam szerzett tapasztalatok alapján az alacsony prioritású problémák nyomon követése támogatta, 
hogy a projekt fókusza a kritikus funkciók biztosításán maradjon, miközben a jövőbeli bővítések előkészítése is folyamatban volt.

%----------------------------------------------------------------------------
\section{Megoldási stratégiák}
%----------------------------------------------------------------------------

A felmerült problémák kezelése során a cél a rendszer stabilitásának, funkcionalitásának és a projekt határidőinek biztosítása volt. 
A problémákra alkalmazott stratégiák strukturáltsága és 
következetessége jelentősen befolyásolja a projekt sikerét \cite{Hajdu2014,Szalay2018,Kaposi2019}.

A technológiai problémák, különösen a Docker-konténerek és a PHP/Twig környezet konfigurációja 
kezdetben instabilitást okoztak, ami futtatási hibákhoz vezetett. Az iteratív 
finomhangolás és a rendszeres tesztelés kulcsfontosságú a szoftver stabilitásának biztosításához \cite{Kovacs2016}. 
A konfigurációk fokozatos javítása, a hibák dokumentálása és a rendszeres tesztkörök alkalmazása jelentősen csökkentette a technikai kockázatot.

A funkcionális kihívások, például a bérléskezelési modulok pontos leképezése a szoftverben, szintén komoly figyelmet igényeltek. 
A komplex üzleti folyamatok modellezése iteratív tesztelés és 
visszacsatolás nélkül gyakran vezet félreértésekhez és hibákhoz \cite{Szalay2018,Kaposi2019}. 
A moduláris fejlesztés és az ismételt tesztelés adott lehetőséget, hogy 
a rendszer logikája a vállalati igényekhez igazodjon, miközben a hibák gyorsan lokalizálhatók és javíthatók voltak.

Az időmenedzsment kritikus tényező volt, mivel a projektet egyedül végeztem. 
A napi és heti ütemtervek kialakítása, valamint a feladatok rangsorolása segítette a munka és a projekt előrehaladásának egyensúlyban tartását.
Az időbeosztás és a feladatpriorizálás kulcsfontosságú a 
projekt határidőn belüli teljesítéséhez és a kiégés elkerüléséhez \cite{Kovacs2016,Hajdu2014}. 
A tervezett és dokumentált ütemezés hozzájárult a hatékony munkavégzéshez és a stressz csökkentéséhez.

A tesztelés és hibajavítás terén a moduláris stratégia alkalmazása biztosította, 
hogy minden komponens külön-külön ellenőrzésen essen át, majd a teljes rendszer integrációját is alaposan teszteltem. 
A folyamatos integráció és a moduláris tesztelés szerepét 
a hibák korai felismerésében és a rendszer megbízhatóságának növelésében \cite{Szalay2018,Kaposi2019}. 
\begin{comment}
%----------------------------------------------------------------------------
\section{Tanulságok a kockázatkezelésből}
%----------------------------------------------------------------------------

A projekt során szerzett tapasztalatok egyértelműen alátámasztották, hogy a kockázatok előrejelzése 
és kezelése alapvető fontosságú az önálló fejlesztések sikeréhez.  
Az előzetesen kialakított kockázatmátrix biztosította a legkritikusabb problémák azonosítását, 
így a figyelmet a rendszer stabilitását leginkább veszélyeztető területekre tudtam összpontosítani, 
miközben a kisebb kockázatok kezelését későbbre halaszthattam.  

A rendszeres tesztelés és részletes dokumentáció elősegítette a hibák gyors felismerését és 
javítását, növelte a fejlesztés átláthatóságát, valamint támogatta a későbbi bővítések gördülékeny megvalósítását.  
A moduláris tesztelési stratégia és az iteratív javítások kombinációja biztosította, hogy a 
rendszer megbízhatóan működjön, miközben a fejlesztési folyamat hatékonyan szervezett maradt.

Az időmenedzsment és a feladatpriorizálás kulcsfontosságú tényezőnek bizonyult a projekt folyamatos előrehaladásában.  
A napi és heti ütemtervek, valamint a rangsorolt feladatlista lehetővé tették, hogy a projekt 
haladása ne veszélyeztesse a munkavégzés minőségét és a határidők betartását.  
A strukturált időbeosztás és a fókuszált munkavégzés döntő jelentőségű a sikeres, önálló fejlesztéshez.
\end{comment}