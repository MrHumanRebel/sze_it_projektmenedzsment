%----------------------------------------------------------------------------
\chapter{\riskproblem}
%---------------------------------------------------------------------------- 

A \textbf{TeDeRMS} projekt során a kockázatkezelés kulcsfontosságú szerepet játszott, 
mivel a projekt önálló fejlesztésként valósult meg, így minden döntés és probléma gyors és hatékony kezelést igényelt. 
A kockázatkezelés célja a potenciális problémák előrejelzése, azok hatásának minimalizálása, 
valamint a projekt sikerének biztosítása volt.

%----------------------------------------------------------------------------
\section{Felmerült problémák}
%----------------------------------------------------------------------------

A fejlesztés során számos kihívás és nehézség merült fel, amelyek kezelése kulcsfontosságú volt a 
projekt sikeréhez és a rendszer funkcionalitásának biztosításához. 
A tapasztalatok alapján megfigyelhető, hogy a projektmenedzsment klasszikus fázisai jól előre jelzik a 
problémák jellegét, de a gyakorlatban gyakran szükség van rugalmasságra és kreatív megoldásokra is \cite{Hajdu2014,Szalay2018}.

\begin{itemize}
    \item \textbf{Technológiai problémák:} a rendszer stabilitása és a Docker-konténerek kezdeti konfigurációja nem volt optimális, ami futtatási hibákhoz vezetett.
\end{itemize}

Ez a probléma rámutatott arra, hogy a modern szoftverfejlesztési környezetekben a technológiai infrastruktúra 
optimalizálása legalább olyan fontos, mint a szoftver logikájának megtervezése. 
A stabil technológiai alap kritikus a rendszer megbízhatósága szempontjából, 
és a kezdeti hibák megfelelő dokumentációval és iteratív konfigurálással kezelhetők \cite{Kovacs2016,Kaposi2019}. 
Saját tapasztalatom szerint az ilyen problémák korai felismerése jelentősen csökkentette a későbbi üzemeltetési nehézségeket.

\begin{itemize}
    \item \textbf{Funkcionális kihívások:} a bérléskezelési folyamatok pontos leképezése a szoftverben nehézkes volt, különösen a különböző státuszok és jogosultságok kezelése.
\end{itemize}

A funkcionalitás pontos implementálása során szembesültem azzal, hogy a vállalati folyamatok összetettsége 
gyakran túlmutat a papíron megadott szabályokon. A rendszertervezés során a folyamatok 
részletes feltérképezése és a valós működés elemzése elengedhetetlen a hibák minimalizálásához \cite{Szalay2018,Hajdu2014}. 
Megfigyeléseim alapján a moduláris architektúra bevezetése és a folyamatok iteratív tesztelése kulcsfontosságú volt a pontos implementáció eléréséhez.

\begin{itemize}
    \item \textbf{Időmenedzsment:} a projekt önálló végzése miatt a napi munka és a projekt előrehaladása 
    közötti egyensúly fenntartása komoly kihívást jelentett.
\end{itemize}

Az időmenedzsment kérdése kiemelten fontos volt, mivel a korlátozott erőforrások és a napi feladatok összehangolása 
folyamatos priorizálást igényelt. A projekt időbeli koordinációja, 
a mérföldkövek betartása és a személyes produktivitás optimalizálása kritikus tényezők az önálló fejlesztések során \cite{Kaposi2019,Kovacs2016}. 
Értékelésem alapján a heti tervezés és a feladatok rendszeres felülvizsgálata segített a haladás fenntartásában.

\begin{itemize}
    \item \textbf{Tesztelés és hibajavítás:} a rendszer moduláris felépítése miatt a hibák lokalizálása és javítása több iterációt igényelt.
\end{itemize}

A hibák azonosítása és javítása a moduláris felépítés miatt komplex feladat volt, mivel egy-egy modul problémája 
láncreakcióként hatott a rendszer más részeire. A folyamatos unit- és integrációs 
tesztelés, valamint a hibák dokumentálása és visszacsatolása alapvető a projekt stabilitása szempontjából \cite{Hajdu2014,Szalay2018}. 
Személyes véleményem szerint az iteratív hibajavítás és a részletes tesztelési protokollok alkalmazása lehetővé 
tette a moduláris rendszer hatékony működését, miközben a tanult módszerek későbbi projektekben is alkalmazhatók.

%----------------------------------------------------------------------------
\section{Kockázatok azonosítása és priorizálása}
%----------------------------------------------------------------------------

A projekt kezdeti fázisában kiemelten fontos volt a potenciális problémák előrejelzése és kezelése, 
mivel a kockázatok megfelelő azonosítása és priorizálása közvetlen hatással van a projekt sikerére 
és a rendszer minőségére. A kockázatmenedzsment alapja a kockázatok strukturált feltérképezése, 
az értékelésük és a megfelelő megelőző intézkedések kialakítása \cite{Hajdu2014,Szalay2018,Kovacs2016}.

A projekt során egy \textbf{kockázatmátrix} került kialakításra, amely azonosította a legkritikusabb problémákat és segítette a megfelelő prioritások meghatározását:

\begin{itemize}
    \item \textbf{Magas prioritású kockázatok:} kritikus hibák a backend működésében, adatvesztés, biztonsági hiányosságok.
\end{itemize}

Ezek a kockázatok a rendszer alapvető működését fenyegették, ezért azonnali figyelmet igényeltek. 
A magas prioritású problémák kezelése kulcsfontosságú a 
rendszer stabilitásának és megbízhatóságának biztosításához \cite{Kaposi2019,Szalay2018}. Saját tapasztalatom alapján a kritikus 
hibák korai azonosítása és a preventív intézkedések bevezetése jelentősen csökkentette a rendszer működési kockázatát.

\begin{itemize}
    \item \textbf{Közepes prioritású kockázatok:} kisebb felhasználói felületbeli problémák, riportálási hibák.
\end{itemize}

A közepes prioritású kockázatok a felhasználói élményt és a riportok pontosságát érintették. 
A szakirodalom kiemeli, hogy ezek a problémák bár nem veszélyeztetik közvetlenül a rendszer működését, 
hosszú távon befolyásolhatják az elfogadást és a felhasználói elégedettséget \cite{Hajdu2014,Kovacs2016}. 
Megítélésem szerint ezeknek a hibáknak a folyamatos monitorozása és iteratív javítása lehetővé tette 
a felhasználói élmény optimalizálását anélkül, hogy a kritikus funkciókat veszélyeztettük volna.

\begin{itemize}
    \item \textbf{Alacsony prioritású kockázatok:} kisebb esztétikai hibák vagy későbbi bővítési igények.
\end{itemize}

Az alacsony prioritású kockázatok elsősorban a rendszer megjelenését vagy a jövőbeli fejlesztések 
ütemezését érintették. A szakirodalom hangsúlyozza, hogy ezek a kockázatok ugyan kevésbé sürgetők, 
de hosszú távon hozzájárulnak a rendszer karbantarthatóságához és a fejlesztési terv fenntarthatóságához \cite{Szalay2018,Kaposi2019}. 
Általam szerzett tapasztalatok alapján az alacsony prioritású problémák nyomon követése lehetővé tette, hogy 
a projekt fókusza a kritikus funkciók biztosításán maradjon, miközben a jövőbeli bővítések előkészítése is folyamatban volt.

%----------------------------------------------------------------------------
\section{Megoldási stratégiák}
%----------------------------------------------------------------------------

A felmerült problémák kezelése során a cél a rendszer stabilitásának, funkcionalitásának és a projekt határidőinek biztosítása volt. 
A problémákra alkalmazott stratégiák strukturáltsága és 
következetessége jelentősen befolyásolja a projekt sikerét \cite{Hajdu2014,Szalay2018,Kaposi2019}.

A technológiai problémák, különösen a Docker-konténerek és a PHP/Twig környezet konfigurációja 
kezdetben instabilitást okoztak, ami futtatási hibákhoz vezetett. Az iteratív 
finomhangolás és a rendszeres tesztelés kulcsfontosságú a szoftver stabilitásának biztosításához \cite{Kovacs2016}. 
A konfigurációk fokozatos javítása, a hibák dokumentálása és a rendszeres tesztkörök alkalmazása jelentősen csökkentette a technikai kockázatot.

A funkcionális kihívások, például a bérléskezelési modulok pontos leképezése a szoftverben, szintén komoly figyelmet igényeltek. 
A komplex üzleti folyamatok modellezése iteratív tesztelés és 
visszacsatolás nélkül gyakran vezet félreértésekhez és hibákhoz \cite{Szalay2018,Kaposi2019}. 
A moduláris fejlesztés és az ismételt tesztelés lehetővé tette, hogy 
a rendszer logikája a vállalati igényekhez igazodjon, miközben a hibák gyorsan lokalizálhatók és javíthatók voltak.

Az időmenedzsment kritikus tényező volt, mivel a projektet egyedül végeztem. 
A napi és heti ütemtervek készítése, valamint a feladatok priorizálása lehetővé tette a munka és a projekt előrehaladásának kiegyensúlyozását. 
Az időbeosztás és a feladatpriorizálás kulcsfontosságú a 
projekt határidőn belüli teljesítéséhez és a kiégés elkerüléséhez \cite{Kovacs2016,Hajdu2014}. 
A tervezett és dokumentált ütemezés hozzájárult a hatékony munkavégzéshez és a stressz csökkentéséhez.

A tesztelés és hibajavítás terén a moduláris stratégia alkalmazása biztosította, 
hogy minden komponens külön-külön ellenőrzésen essen át, majd a teljes rendszer integrációját is alaposan teszteltem. 
A folyamatos integráció és a moduláris tesztelés szerepét 
a hibák korai felismerésében és a rendszer megbízhatóságának növelésében \cite{Szalay2018,Kaposi2019}. 
A megközelítés lehetővé tette a hibák gyors javítását és a stabil rendszer kialakítását, minimalizálva a későbbi problémákat.

%----------------------------------------------------------------------------
\section{Tanulságok a kockázatkezelésből}
%----------------------------------------------------------------------------

A projekt során szerzett tapasztalatok világosan rámutattak arra, hogy a kockázatkezelés elengedhetetlen eleme az önálló fejlesztéseknek. 
A problémák előrejelzése és a priorizálás nem csupán a projekt sikerességét növeli, 
hanem lehetővé teszi a fejlesztési folyamat hatékony szervezését és a kritikus hibák minimalizálását \cite{Hajdu2014,Kovacs2016,Szalay2018}. 
Az előzetesen kialakított kockázatmátrix segített azonosítani a legkritikusabb területeket, 
így a figyelmet a legfontosabb problémákra tudtam összpontosítani, miközben a kisebb kockázatok későbbi kezelésre vártak.

A rendszeres tesztelés és a dokumentáció készítése szintén kulcsfontosságú volt. 
A moduláris tesztelés, az iteratív javítások és a 
részletes dokumentáció lehetővé teszik a hibák gyors felismerését és hatékony javítását, 
ami jelentősen csökkenti a rendszerhibák előfordulását \cite{Kaposi2019,Szalay2018}. 
Saját tapasztalatom szerint a folyamatos tesztelés és dokumentáció nemcsak a rendszer 
stabilitását biztosította, hanem a projekt átláthatóságát is növelte, így a későbbi bővítések is könnyebben megvalósíthatók voltak.

Az időmenedzsment és a feladatpriorizálás szintén kiemelt jelentőségűnek bizonyult. 
Az önálló fejlesztés során a napi és heti ütemtervek, valamint a feladatok rangsorolása 
lehetővé tette a projekt folyamatos előrehaladását anélkül, hogy a munkavégzés minősége vagy a határidők veszélybe kerültek volna \cite{Kovacs2016,Hajdu2014}. 
Saját értékelésem szerint a strukturált időbeosztás és a fókuszált feladatvégzés elengedhetetlen a sikeres, önálló projektvégrehajtáshoz.