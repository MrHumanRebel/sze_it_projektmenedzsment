%----------------------------------------------------------------------------
\chapter{\riskproblem}
%----------------------------------------------------------------------------

A kockázatkezelést már a projektindításkor rendszerbe szerveztem, mert egyszemélyes fejlesztésnél minden késedelem közvetlenül az átfutási időt veszélyezteti \cite{Hajdu2014}. A kockázatmátrixot a szakirodalom javaslatai szerint a valószínűség és hatás alapján töltöttem ki, majd minden tételhez megelőző és reagáló intézkedést rendeltem \cite{Kovacs2016,Kaposi2019}.

\section{Főbb kockázatok és válaszok}

\begin{table}[H]
    \centering
    \renewcommand{\arraystretch}{1.2}
    \begin{tabular}{@{}p{4cm} p{2cm} p{2cm} p{5cm}@{}}
        \toprule
        \textbf{Kockázat} & \textbf{Valószínűség} & \textbf{Hatás} & \textbf{Válasz}\\
        \midrule
        Docker/Nginx konfigurációs hiba & Közepes & Magas & Előre összeállított Compose profilok, automatikus healthcheck-ek, rollback script.\\
        Adatvesztés vagy sérülés & Alacsony & Magas & Napi dump cron, verziózott migrációk, tesztadatbázis bevezetése.\\
        Funkcionális félreértelmezés (bérlés státuszok) & Közepes & Közepes & Részletes use case leírások, prototípus-demó minden sprint végén.\\
        Időcsúszás az önálló fejlesztés miatt & Magas & Közepes & Heti 25 órás keret, buffer a kockázatos modulokra, scope-fagyasztás a 3. sprint után.\\
        Jogosultsági hibák vagy jogosulatlan hozzáférés & Alacsony & Magas & Role-based access control, kötelező audit log, automatizált belépési tesztek.\\
        Teljesítmény problémák nagy eszközlistán & Közepes & Közepes & Indexelt adatbázis mezők, lapozás és gyorsszűrés beépítése, célzott terheléses tesztek.\\
        Külső szolgáltatás (Google OAuth) elérhetetlensége & Alacsony & Közepes & Lokális fiók fallback, auth endpoint monitoring, riasztás beállítása.\\
        \bottomrule
    \end{tabular}
    \caption{Kockázatkezelési mátrix a TeDeRMS projekthez. \footnotesize Forrás: saját szerkesztés, projekt dokumentáció}
    \label{tab:risk_matrix}
\end{table}

\section{Problémák kezelése a gyakorlatban}

\textbf{Technológiai kockázatok:} a konténerindításkor jelentkező hibákra automatizált ellenőrző szkripteket írtam, így a futási problémák már a fejlesztői környezetben feltűntek. Ez összhangban van a megelőző kontrollokra építő ajánlással \cite{Szalay2018}.

\textbf{Funkcionális eltérések:} a bérlés státuszmátrixa többször pontosításra szorult. Minden módosítás után rövid használhatósági tesztet futtattam a vállalat kulcsfelhasználóival, és az eredményeket jegyzőkönyvben rögzítettem.

\textbf{Idő- és erőforráskockázat:} ha a tényleges ráfordítás meghaladta a tervezettet, a következő sprintben kevesebb új funkciót vállaltam, és a hibajavításra fókuszáltam. A módszer megfelelt a szakirodalomban javasolt gördülő tervezésnek \cite{Kaposi2019}.

\begin{figure}[H]
    \centering
    \includegraphics[width=80mm, keepaspectratio]{figures/problem.jpg}
    \caption{Problémamegoldási folyamat vizualizációja. \footnotesize Forrás: saját szerkesztés, projekt dokumentáció}
    \label{fig:problem_solving}
\end{figure}
