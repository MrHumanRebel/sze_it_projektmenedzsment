%----------------------------------------------------------------------------
\chapter{\projectoverview}
%----------------------------------------------------------------------------
\section{A vállalat és az üzleti igények}

A TéDé Rendezvények rendezvénytechnikai szolgáltatóként magas értékű eszközparkot és párhuzamos projekteket kezel. Saját tapasztalatom szerint a papíralapú és Excel-folyamatok lassú egyeztetést, többszörös adatbevitelt és hibás készletinformációt eredményeztek. A szakirodalom is hangsúlyozza, hogy az átlátható erőforrás- és dokumentumkezelés kulcsa az integrált rendszerbe szervezett folyamat \cite{Kovacs2016,Kaposi2019}. A projekt ezért egy egységes, online bérlés- és projektkezelő platform létrehozását tűzte ki célul.

A legfontosabb üzleti elvárások:
\begin{itemize}
    \item pontos készlet- és eszköznyilvántartás valós idejű státuszokkal;
    \item gyors ajánlatadás és projektindítás automatizált sablonokkal;
    \item jogosultságkezelés és naplózás a felelősségi körök elkülönítésére;
    \item egységes, auditálható dokumentumtár a rendezvénydokumentációhoz.
\end{itemize}

\section{Funkciók rövid áttekintése}

A TeDeRMS moduláris felépítésben készült, hogy a vállalati folyamatokat közvetlenül támogassa.
\begin{itemize}
    \item \textbf{Hitelesítés és onboarding:} helyi fiók vagy Google OAuth2, majd csatlakozó kód a vállalati tenant-hoz. A megoldás megfelel a szakirodalom által kiemelt jogosultsági kontrollnak \cite{Hajdu2014}.
    \item \textbf{Dashboard:} valós idejű eszköz-, projekt- és felhasználói mutatók, valamint közös naptár nézet a kapacitástervezéshez \cite{Szalay2018}.
    \item \textbf{Eszközkezelés:} címkézett eszközök, részletes metaadatok, fotók és bérlési árak; rugalmas szűrés projektre, kategóriára és státuszra.
    \item \textbf{Projektlap:} ügyféladatok, felelősök, helyszínek, státuszok és erőforrás-hozzárendelés egy nézetben, hogy a tervezés és kontroll összekapcsolódjon \cite{Kaposi2019}.
    \item \textbf{Riportálás:} alap pénzügyi és kihasználtsági riportok, amelyek a lezárt projektekből generálhatók.
\end{itemize}

\begin{figure}[H]
    \centering
    \includegraphics[width=110mm, keepaspectratio]{figures/login.jpg}
    \caption{Bejelentkezés és csatlakozás a vállalathoz. \footnotesize Forrás: saját szerkesztés, vállalati demóképernyő}
    \label{fig:login}
\end{figure}

\begin{figure}[H]
    \centering
    \includegraphics[width=110mm, keepaspectratio]{figures/dashboard.jpg}
    \caption{Dashboard a fő erőforrásmutatókkal. \footnotesize Forrás: saját szerkesztés, vállalati demóképernyő}
    \label{fig:dashboard}
\end{figure}

\section{Technológiai háttér}

A rendszer fejlesztése a következő technológiákra épül, amelyek a skálázhatóságot és a könnyű üzemeltetést szolgálják:
\begin{itemize}
    \item \textbf{Backend:} PHP és Twig sablonmotor a gyors szerveroldali rendereléshez.
    \item \textbf{Frontend:} reszponzív nézetek és újrafelhasználható komponensek.
    \item \textbf{Adatbázis:} MySQL a strukturált adatokhoz, migrációs scriptekkel.
    \item \textbf{Konténerizálás:} Docker Compose lokális és éles környezethez, előre definiált \texttt{.env} változókkal.
    \item \textbf{Reverse proxy:} Nginx a TLS-védelemmel ellátott forgalom és a statikus fájlok kiszolgálására.
    \item \textbf{Verziókezelés:} Git alapú munkafolyamat, hogy a változások visszakövethetők legyenek \cite{Kovacs2016}.
\end{itemize}
