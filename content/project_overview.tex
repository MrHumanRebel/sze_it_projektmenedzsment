%----------------------------------------------------------------------------
\chapter{\projectoverview}
%----------------------------------------------------------------------------
\section{A vállalat és az üzleti igények}

A projekt célja a \textbf{TéDé Rendezvények} vállalat bérlés és projektkezelési folyamatainak teljes digitalizálása volt. 
A korábbi papíralapú és Excel-alapú folyamatok nem feleltek meg a vállalat növekvő igényeinek,
és egy modern, integrált rendszer kialakítása vált szükségessé. Az online elérhető üzenetküldő és felhő alapú megoldások
nem kínáltak megfelelő rugalmasságot és testreszabhatóságot, ezért egy egyedi fejlesztésű rendszer mellett született döntés.

A vállalat számára kiemelten fontos volt egy egységes, modern és felhasználóbarát rendszer, amely:
\begin{itemize}
    \item automatizálja a bérlési folyamatokat,
    \item nyomon követi a készleteket,
    \item biztosítja az adminisztráció teljes körű kezelését,
    \item egyszerűsíti a munkafolyamatokat,
    \item csökkenti a hibalehetőségeket,
    \item javítja a kommunikációt a csapaton belül,
    \item egységesíti a dokumentációt,
    \item minden információ egy helyen elérhető,
    \item lehetővé teszi a gyors árajánlatkészítést.
\end{itemize}

%----------------------------------------------------------------------------
\section{A projekt célkitűzései}
%----------------------------------------------------------------------------

További célok voltak:
\begin{itemize}
    \item modern, felhasználóbarát felület
    \item egyszerűen telepíthető és skálázható \textbf{Docker} segítségével
    \item könnyen karbantartható és bővíthető architektúra
    \item biztonságos hozzáférés-kezelés és jogosultságok
    \item részletes riportálási és statisztikai funkciók
\end{itemize}

%----------------------------------------------------------------------------
\section{A projekt összetettsége és kihívásai}
%----------------------------------------------------------------------------

A projekt felettébb összetettnek tekinthető, mivel:
\begin{itemize}
    \item többféle felhasználói szerepkört kellett kezelnie (admin, munkatárs, menedzser),
    \item integrálni kellett különböző adatforrásokat és a bérlési folyamatokat,
    \item biztonsági és hozzáférés-kezelési követelményeknek kellett megfelelnie,
    \item a fejlesztés során több technológiát kellett összehangolni a rugalmasság és megbízhatóság érdekében.
\end{itemize}

%----------------------------------------------------------------------------
\section{Technológiai háttér}
%----------------------------------------------------------------------------

A rendszer fejlesztése a következő technológiákra épült:
\begin{itemize}
    \item \textbf{Backend:} PHP a \textbf{Twig} sablonmotorral,
    \item \textbf{Frontend:} modern responsive felhasználói felület Twig sablonokkal,
    \item \textbf{Konténerizálás és telepítés:} Docker és Docker Compose, előre elkészített konténerekben,
    \item \textbf{Konfiguráció:} testreszabható \texttt{.env} fájlok segítségével.
    \item \textbf{Reverse proxy:} Nginx a kérések kezelésére és a statikus fájlok kiszolgálására,
    \item \textbf{Adatbázis:} MySQL a megbízható adatkezelés érdekében,
    \item \textbf{Verziókezelés:} Git a kód nyomon követésére és együttműködésre.
\end{itemize}

Ez a technológiai kombináció biztosítja a rendszer gyors telepítését, stabil működését és könnyű bővíthetőségét.