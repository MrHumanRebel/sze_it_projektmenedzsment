%----------------------------------------------------------------------------
\chapter{\bevezetes}
%---------------------------------------------------------------------------------------------------------------------------------------------

A mai vállalati világban az idő és az erőforrások hatékony kezelése nem csupán optimális, hanem létfontosságú versenyelőny. 
A hagyományos, papíralapú vagy szigetszerűen kezelt folyamatok lassúak, átláthatatlanok és hibákra hajlamosak. 
A vállalatoknak ezért exponenciálisan igénye van olyan rendszerekre, amelyek gyorsítják a munkafolyamatokat, csökkentik 
a hibalehetőségeket és automatizálják a mindennapi működést.

Ebben a környezetben vállaltam egy önálló projektet a vállalatnak, ahol már évek óta tevékenykedem.
Egy \textbf{egyedi bérléskezelő rendszer (Rental Management System – RMS)} fejlesztését, 
amely a vállalat minden projektéhez kapcsolódó folyamatait digitális formába önti. 
A cél nem csupán a szoftver létrehozása volt, hanem annak teljes életciklusú menedzselése, 
a tervezéstől és ütemezéstől kezdve a kockázatok azonosításán át a megvalósításig és az átadásig és karbantartásig.

A dolgozat célja a projekt részletes bemutatása: ismertetem az RMS céljait, funkcióit és technológiai hátterét. 
Szakmai szempontból külön hangsúlyt kapnak a projektmenedzsment folyamatok, 
különösen a \textbf{tervezés}, az \textbf{ütemezés} és a \textbf{kockázatkezelés}, valamint annak bemutatása, 
hogy ezek miként járultak hozzá a projekt sikeres megvalósításához.

\vspace{1em}
\noindent A dolgozat többek között a következő kérdésekre is választ keres:
\begin{itemize}
    \item Hogyan tervezhető és menedzselhető hatékonyan egy egyedi vállalati szoftverfejlesztési projekt?
    \item Milyen módszerek és eszközök biztosítják az erőforrások optimális felhasználását és a kockázatok minimalizálását?
    \item Milyen szakmai tanulságok vonhatók le az önálló projektmenedzsment gyakorlatából, amelyek más projektekben is alkalmazhatók?
\end{itemize}

\vspace{1em}
\noindent A következő fejezetben bemutatom a projekt konkrét célkitűzéseit és főbb jellemzőit, 
kiemelve, hogy a fejlesztett rendszer milyen problémákat old meg és milyen értéket teremt a vállalat számára.
