%----------------------------------------------------------------------------
\chapter{\bevezetes}
%-------------------------------------------------------------------------------------------------------------------------------

A digitális transzformációban működő vállalatoknál az idő, a költség és a minőség összehangolása döntően projektmenedzsment-feladat, amelyet rendszerszerű tervezés és kockázatkezelés támaszt alá \cite{Hajdu2014,Szalay2018}. A TéDé Rendezvények vállalatnál szerzett többéves tapasztalatom azt mutatta, hogy a papíralapú vagy táblázatos nyilvántartások lassítják a döntéshozatalt és növelik a hibák esélyét, ezért saját fejlesztésben valósítottam meg a \textbf{TeDeRMS} (Rental Management System) megoldást.

A dolgozat célja, hogy röviden bemutassa a rendszer funkcióit és technológiai hátterét, miközben a projektmenedzsment elméleti kereteit végig összekapcsolja a gyakorlati megvalósítással. A hangsúly azokon az elemein van, amelyek a vállalati környezetben bizonyítottan értéket teremtettek: az igényfelmérésen, az ütemezésen, a kockázatkezelésen és a bevezetésen \cite{Kovacs2016,Kaposi2019}.

A fő kérdések, amelyekre a fejezetek választ adnak:
\begin{itemize}
    \item Milyen üzleti igényekre épült a rendszer, és ezek hogyan jelennek meg a funkcionális tervezésben?
    \item Hogyan alkalmaztam a projektéletciklus fázisait a TeDeRMS fejlesztésére, és milyen mérőszámok jelezték a haladást?
    \item Mely kockázatok befolyásolták leginkább az önálló fejlesztést, és milyen válaszokat adtam rájuk a szakirodalom ajánlásai alapján?
    \item Milyen tanulságokat lehet más, hasonló vállalati fejlesztésekre átültetni?     
\end{itemize}
